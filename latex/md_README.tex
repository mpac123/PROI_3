\section*{Polecenie}

Menu programu okienkowego jest kolekcją wyborów. Każdy z wyborów jest albo jednoznaczny albo wskazuje na inne menu. Oprogramować bibliotekę klas reprezentujących menu. Biblioteka ma umożliwiać następujące czynności\-:
\begin{DoxyItemize}
\item Dodawanie/usuwanie wyborów jednoznacznych do/z (pod)menu.
\item Dodawanie/usuwanie podmenu do/z (pod)menu.
\item Przypisywanie wyborów jednoznacznych do pewnych funkcji.
\item Rozwijanie/zwijanie podmenu.
\item Dokonywanie wyborów jednoznacznych. Napisać program, który wszystkie powyższe funkcjonalności udostępnia poprzez polecenia wydawane z klawiatury. Przyjąć, że za wyborami jednoznacznymi może kryć się stały zbiór funkcji wypisujący na konsoli komunikaty typu “\-Zadziałała funkcja nr 14”.
\end{DoxyItemize}

\section*{Założenia}


\begin{DoxyEnumerate}
\item Menu zawsze musi się składać z conajmniej jednego Wyboru typu \hyperlink{classPodmenu}{Podmenu}. Nie można usunąć takiego \hyperlink{classPodmenu}{Podmenu}.
\item Usunięcie \hyperlink{classPodmenu}{Podmenu} oznacza usunięcie razem z nim wszystkich zagnieżdżonych w nim Wyborów.
\item Po usunięciu elementu (elementów) Menu kursor wskazuje na poprzedni obiekt.
\item Po dodaniu nowego zagnieżdżonego Wyboru, kursos jest ustawiony na tym nowododanym Wyborze.
\item Jeżeli \hyperlink{classPodmenu}{Podmenu} jest zwinięte, nie ma możliwości przejścia kursorem do któregokolwiek z zagnieżdżonych w nim Wyborów. W tym celu należy je najpierw rozwinąć.
\end{DoxyEnumerate}

\section*{Struktura klas reprezentujących Menu}

Klasy użyte do zaprojektowania Menu\-:


\begin{DoxyEnumerate}
\item \hyperlink{classWybor}{Wybor}
\begin{DoxyItemize}
\item \hyperlink{classJednoznaczny}{Jednoznaczny}
\item \hyperlink{classPodmenu}{Podmenu}
\end{DoxyItemize}
\item \hyperlink{classKolekcja}{Kolekcja}
\item \hyperlink{classObsluga}{Obsluga}
\item \hyperlink{classInterfejs}{Interfejs}
\end{DoxyEnumerate}

\section*{Metody wykorzystywane w interfejsie użytkownika}

Użytkownik dysponuje wieloma metodami, pozwalającymi na swobodne poruszanie się po Menu i zarządzanie nim. Są to metody publiczne klasy \hyperlink{classObsluga}{Obsluga} -\/ w sekcji tej klasy szczegółowo opisano działanie i parametry każdej z metod.

\section*{Sposób testowania}

Za obsługę interfejsu tekstowego służącego do testowania programu odpowiada klasa \hyperlink{classInterfejs}{Interfejs} -\/ szczegółowy opis klasy wraz z opisem jej metod publicznych znajduje się w osobnej sekcji poświęconej tej klasie.

\subsection*{Sposób 1 -\/ od zera}

Po uruchomieniu programu użytkownik zostaje poproszony o podanie nazwy Menu -\/ zostanie utworzone pierwsze \hyperlink{classPodmenu}{Podmenu} o stopniu zagnieżdżenia 0. Następnie uzytkownik może poruszać się po Menu za pomocą klawiszy W (góra) i S (dół). Dodatkowe opcje są zależne od tego, czy kursos pokazuje na \hyperlink{classPodmenu}{Podmenu} czy Wybór \hyperlink{classJednoznaczny}{Jednoznaczny} i wszystkie możliwe do użycia skróty wraz z opisami są wypisane na ekranie bezpośrednio pod Menu. Taki sposób testowania umożliwia metoda publiczna klasy \hyperlink{classInterfejs}{Interfejs} Start().

\subsection*{Sposób 2 -\/ bazując na stworzonym Menu}

Program można również uruchomić dla stworzonego wcześniej w interfejsie użytkownika Menu. Wówczas użytkownik nie musi wprowadzać nazwy Menu -\/ od razu można nawigować i zarządzać Menu w taki sam sposób, jak opisano dla sposobu 1. Taki sposób testowania umozliwia metoda publiczna klasy \hyperlink{classInterfejs}{Interfejs} Program(\-Obsluga \&\-M), gdzie obiekt M został wcześniej stworzony (i modyfikowany). 