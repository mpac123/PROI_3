\section*{Polecenie}

Menu programu okienkowego jest kolekcją wyborów. Każdy z wyborów jest albo jednoznaczny albo wskazuje na inne menu. Oprogramować bibliotekę klas reprezentujących menu. Biblioteka ma umożliwiać następujące czynności\-:
\begin{DoxyItemize}
\item Dodawanie/usuwanie wyborów jednoznacznych do/z (pod)menu.
\item Dodawanie/usuwanie podmenu do/z (pod)menu.
\item Przypisywanie wyborów jednoznacznych do pewnych funkcji.
\item Rozwijanie/zwijanie podmenu.
\item Dokonywanie wyborów jednoznacznych. Napisać program, który wszystkie powyższe funkcjonalności udostępnia poprzez polecenia wydawane z klawiatury. Przyjąć, że za wyborami jednoznacznymi może kryć się stały zbiór funkcji wypisujący na konsoli komunikaty typu “\-Zadziałała funkcja nr 14”.
\end{DoxyItemize}

\section*{Założenia}


\begin{DoxyEnumerate}
\item Menu zawsze musi się składać z conajmniej jednego Wyboru typu \hyperlink{classPodmenu}{Podmenu}. Nie można usunąć takiego \hyperlink{classPodmenu}{Podmenu}.
\item Usunięcie \hyperlink{classPodmenu}{Podmenu} oznacza usunięcie razem z nim wszystkich zagnieżdżonych w nim Wyborów.
\item Po usunięciu elementu (elementów) Menu kursor wskazuje na następny obiekt.
\item Jeżeli \hyperlink{classPodmenu}{Podmenu} jest zwinięte, nie ma możliwości przejścia kursorem do któregokolwiek z zagnieżdżonych w nim Wyborów. W tym celu należy je najpierw rozwinąć.
\end{DoxyEnumerate}

\section*{Struktura klas reprezentujących Menu}

Lista klas w kolejności nadrzędności, tzn. klasa wymieniona jako pierwsza nie korzysta z żadnej z innych klas, jest za to wykorzystywana przez klasy nadrzędne, w szczególności przez klasę bezpośrednio nadrzędną, tzn. klasę o numerze 2.


\begin{DoxyEnumerate}
\item \hyperlink{classWybor}{Wybor}
\begin{DoxyItemize}
\item \hyperlink{classJednoznaczny}{Jednoznaczny}
\item \hyperlink{classPodmenu}{Podmenu}
\end{DoxyItemize}
\item \hyperlink{classKolekcja}{Kolekcja}
\item \hyperlink{classObsluga}{Obsluga}
\item \hyperlink{classInterfejs}{Interfejs} 
\end{DoxyEnumerate}